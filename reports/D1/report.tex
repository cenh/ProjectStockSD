\documentclass[11pt]{article}
\usepackage[a4paper, hmargin={2.8cm, 2.8cm}, vmargin={2.5cm, 2.5cm}]{geometry}
\usepackage{eso-pic} % \AddToShipoutPicture
\usepackage{graphicx} % \includegraphics
\usepackage[utf8]{inputenc}
\usepackage[danish]{babel}
\usepackage[T1]{fontenc}
\usepackage{hyperref}
\usepackage{amsmath, amscd}
\usepackage{amsmath,amscd}
\usepackage{amssymb}
\usepackage{amsthm}
\usepackage{enumerate}
\usepackage{graphicx}
\usepackage{framed}
\usepackage{color}
%% Change `ku-farve` to `nat-farve` to use SCIENCE's old colors or
%% `natbio-farve` to use SCIENCE's new colors and logo.
\def \ColourPDF {../include/ku-farve}

%% Change `ku-en` to `nat-en` to use the `Faculty of Science` header
\def \TitlePDF {../include/ku-en}  % University of Copenhagen

\title{
  \vspace{3cm}
  \Huge{D1} \\
  \Large{Software Udvikling 2016}
}

\author{
	\Large{Stefan Friis Tofte} - \textbf{jwr342}% - \texttt{stefan.f.tofte@gmail.com}
	\and
	\Large{Mads Kronborg} - \textbf{xlq446}% - \texttt{kronborg96@gmail.com}
	\and
	\Large{Lasse Halberg Haarbye} - \textbf{lpt113}% - \texttt{ninjalf2@gmail.com}
	\and
	\Large{Christian E.N. Hansen} - \textbf{vmk541}% - \texttt{cralle@outlook.com}
}

\date{
    \today
}

\begin{document}


\AddToShipoutPicture*{\put(0,0){\includegraphics*[viewport=0 0 700 600]{\ColourPDF}}}
\AddToShipoutPicture*{\put(0,602){\includegraphics*[viewport=0 600 700 1600]{\ColourPDF}}}

\AddToShipoutPicture*{\put(0,0){\includegraphics*{\TitlePDF}}}

\clearpage\maketitle
\thispagestyle{empty}

\newpage
\tableofcontents
\newpage
\section{Afklaring af krav}
\subsection{Krav}
\begin{enumerate}
\item
\item
\end{enumerate}
\subsection{Interessenter}
\begin{enumerate}
\item Alexander, studerende
\item Jyrki, koordinator
\item Business Club
\item Instruktor/Lærere X
\item Jette, sekretær
\end{enumerate}
\subsection{Brugsscenarier}

\begin{enumerate}
  \item Systemet skal kunne vise projekter knyttet til specifikke \textit{keywords}.
  \item Det skal være muligt at se et katalog over projekter.
  \item Systemet skal kunne præsentere \textit{up-to-date} oplysninger om potentielle vejledere, herunder:
  \begin{itemize}
    \item Tidligere projekter.
    \item Forskningspublikationer.
    \item Kontaktoplysninger.
  \end{itemize}
  \item Systemet skal vise en status på en vejleders arbejdsbyrde \textit{up-to-date} (hvad der præcist menes med dette, skal yderlige specificeres).
  \item En vejleder eller en person knyttet til en \textit{business-club}, skal kunne tilføje et projekt til projektkataloget.
\end{enumerate}

\begin{itemize}
\item Titel:
  \begin{itemize}
  \item Primær aktor:
  \item Scope:
  \item Niveau:
  \item Scenarie/historie:
  \end{itemize}
\end{itemize}
\section{Opsætning af udviklingsmiljø}
\subsection{Git}
Vi har valgt at bruge Git som vores versionskontrol værktøj, og bruger GitHub som webplatform til netop dette.
Vi har opsat et 'repository', som bliver delt op i flere sektioner. Vi har en forside, som består af en 'README' fil, hvori vi deler materialer/resurser som kan være relevante for vores projekt.
Derudover har vi opsat afsnit til rapporter, og til selve koden. \\
Vores repository har på nuværende tidspunkt følgende mappe-struktur:
\begin{itemize}
\item reports
  \begin{itemize}
  \item D0
  \item D1
  \item include
  \end{itemize}
\item site
  \begin{itemize}
  \item backend
  \item include
  \end{itemize}
\end{itemize}
\subsection{Programmeringssprog/IDE}
Vi udvikler en backend i python, hvis formål er at hente relevante .html sider og derefter høste dem for information vi skal bruge. Hvad end relevant information programmet finder, skal gemmes i form af xml, en sql database eller andet, dette er på nuværende tidspunkt ikke fastlagt. \\
Vi udvikler ikke i en bestemt IDE, men derimod i vores inviduelle præference. Dette ville typisk være Emacs eller Atom.
\section{Udforskning af teknologien}

Til vores prototype har vi lavet en simpel scraper, der udtrækker information fra hjemmesiden: \href{http://www.diku.dk/Ansatte}{DIKU Ansatte}, og finder information udfra html-tags.

\section{Planlægning af næste iteration}
I det kommende \textit{sprint} vil vi fokusere på at høstning af data. Vi vurderer at det er en kerneopgave i systemet, at automatisk opsamle \textit{up-to-date} data. Vi skal afgøre hvilke informationer, om de enkelte vejledere, der relevante og mulige at præsentere for brugerne.
Vi skal overveje og designe en måde man kan verificere person på.
Næste iteration vil altså primært bestå af udvikling af prototype baseret på vores brugsscenarier samt dokumentation af samme. Vi vil se nærmere på vores scraper, da den skal bruges til at indsamle information til alle scenarierne.
\end{document}
