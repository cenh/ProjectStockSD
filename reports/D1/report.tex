\documentclass[11pt]{article}
\usepackage[a4paper, hmargin={2.8cm, 2.8cm}, vmargin={2.5cm, 2.5cm}]{geometry}
\usepackage{eso-pic} % \AddToShipoutPicture
\usepackage{graphicx} % \includegraphics
\usepackage[utf8]{inputenc}
\usepackage[danish]{babel}
\usepackage[T1]{fontenc}

% \Packages for amsmath
\usepackage{amsmath, amscd}
\usepackage{amsmath,amscd}
\usepackage{amssymb}
\usepackage{amsthm}
\usepackage{enumerate}
\usepackage{graphicx}
\usepackage{framed}
\usepackage{color}
%% Change `ku-farve` to `nat-farve` to use SCIENCE's old colors or
%% `natbio-farve` to use SCIENCE's new colors and logo.
\def \ColourPDF {../include/ku-farve}

%% Change `ku-en` to `nat-en` to use the `Faculty of Science` header
\def \TitlePDF {../include/ku-en}  % University of Copenhagen

\title{
  \vspace{3cm}
  \Huge{D1} \\
  \Large{Software Udvikling 2016}
}

\author{
  \Large{Philip Munksgaard} \\
  \texttt{pmunksgaard@gmail.com} \\
}

\date{
    \today
}

\begin{document}


\AddToShipoutPicture*{\put(0,0){\includegraphics*[viewport=0 0 700 600]{\ColourPDF}}}
\AddToShipoutPicture*{\put(0,602){\includegraphics*[viewport=0 600 700 1600]{\ColourPDF}}}

\AddToShipoutPicture*{\put(0,0){\includegraphics*{\TitlePDF}}}

\clearpage\maketitle
\thispagestyle{empty}

\newpage
\tableofcontents
\newpage
\section{Afklaring af krav}
\subsection{Krav}
\begin{enumerate}
\item
\item
\end{enumerate}
\subsection{Interessenter}
\begin{enumerate}
\item Alexander, studerende
\item Jyrki, koordinator
\item Business Club
\item Instruktor/Lærere X
\item Jette, sekretær
\end{enumerate}
\subsection{Brugsscenarier}

\begin{itemize}
  \item Brugeren skal kunne søge efter et specifik emne.
  \item System skal automatisk kunne opsamle \textit{up-to-date} data om potentielle vejledere, herunder:
  \begin{itemize}
    \item Tidligere projekter.
    \item Forskningspublikationer.
    \item Kontaktoplysninger.
    \item
  \end{itemize}
\end{itemize}

\begin{itemize}
\item Titel:
  \begin{itemize}
  \item Primær aktor:
  \item Scope:
  \item Niveau:
  \item Scenarie/historie:
  \end{itemize}
\end{itemize}
\section{Opsætning af udviklingsmiljø}
\subsection{Git}
Vi har valgt at bruge Git som vores versionskontrol værktøj, og bruger GitHub som webplatform til netop dette.
Vi har opsat et 'repository', som bliver delt op i flere sektioner. Vi har en forside, som består af en 'README' fil, hvori vi deler materialer/resurser som kan være relevante for vores projekt.
Derudover har vi opsat afsnit til rapporter, og til selve koden. \\ \\
Vores repository har på nuværende tidspunkt følgende mappe-struktur:
\begin{itemize}
\item reports
  \begin{itemize}
  \item D0
    \begin{itemize}
    \item Team_Contract.tex
    \end{itemize}
  \item D1
    \begin{itemize}
    \item delaflevering-1.pdf
    \item report.tex
    \end{itemize}
  \item include
    \begin{itemize}
    \item ku-en.pdf
    \item ku-farve.pdf
    \item nat-en.pdf
    \item nat-farve.pdf
    \item natbio-farve.pdf
    \end{itemize}
  \end{itemize}
\item site
  \begin{itemize}
  \item backend
    \begin{itemize}
    \item db.sql
    \item scrapter_template.py
    \end{itemize}
  \item include
    \begin{itemize}
    \item style.css
    \end{itemize}
  \end{itemize}
\end{itemize}
\section{Udforskning af teknologien}
\section{Planlægning af næste iteration}
\end{document}
