\documentclass[11pt]{article}
\usepackage[a4paper, hmargin={2.8cm, 2.8cm}, vmargin={2.5cm, 2.5cm}]{geometry}
\usepackage{eso-pic} % \AddToShipoutPicture
\usepackage{graphicx} % \includegraphics
\usepackage[utf8]{inputenc}
\usepackage[danish]{babel}
\usepackage[T1]{fontenc}
\usepackage{hyperref}
\usepackage{amsmath, amscd}
\usepackage{amsmath,amscd}
\usepackage{amssymb}
\usepackage{amsthm}
\usepackage{enumerate}
\usepackage{graphicx}
\usepackage{framed}
\usepackage{color}
\usepackage{listings}
\lstset{
	frame=single,
	breaklines=true,
	postbreak=\raisebox{0ex}[0ex][0ex]{\ensuremath{\color{red}\hookrightarrow\space}}
}
%% Change `ku-farve` to `nat-farve` to use SCIENCE's old colors or
%% `natbio-farve` to use SCIENCE's new colors and logo.
\def \ColourPDF {../include/ku-farve}

%% Change `ku-en` to `nat-en` to use the `Faculty of Science` header
\def \TitlePDF {../include/ku-en}  % University of Copenhagen

\title{
  \vspace{3cm}
  \Huge{D2} \\
  \Large{Software Udvikling 2016}
}

\author{
	\Large{Stefan Friis Tofte} - \textbf{jwr342}% - \texttt{stefan.f.tofte@gmail.com}
	\and
	\Large{Mads Kronborg} - \textbf{xlq446}% - \texttt{kronborg96@gmail.com}
	\and
	\Large{Lasse Halberg Haarbye} - \textbf{lpt113}% - \texttt{ninjalf2@gmail.com}
	\and
	\Large{Christian E.N. Hansen} - \textbf{vmk541}% - \texttt{cralle@outlook.com}
}

\begin{document}


\AddToShipoutPicture*{\put(0,0){\includegraphics*[viewport=0 0 700 600]{\ColourPDF}}}
\AddToShipoutPicture*{\put(0,602){\includegraphics*[viewport=0 600 700 1600]{\ColourPDF}}}

\AddToShipoutPicture*{\put(0,0){\includegraphics*{\TitlePDF}}}

\clearpage\maketitle
\thispagestyle{empty}

\newpage
\tableofcontents
\newpage

\section{Implementering af opgaver}
Vi har for denne iteration, mødtes hver mandag til øvelsestimerne og nogle gange om tirsdagen. Vi har diskuteret hvilke opgaver der var vigtige, og siddet sammen og implementeret visse opgaver i fællesskab. \\ \\
Ved sidste iteration regnede vi med at, vi skulle fokusere på høstning af data til denne iteration. Her har vi ændret holdning, og har istedet for fokuseret på at få sat en \textit{Django-webapplikation} op på en server, så vi har kunne afprøve og se det kører på en rigtig server med tilhørende database. \\ \\
Vi har igennem \textit{Django-frameworket} fået opsat nogle \textit{Models} som indeholder data over projekter, vejleder og andet (Dette bliver beskrevet nærmere i  næste afsnit omkring afprøvning af Django).
Derved har vi opfyldt vores \textit{brugsscenarie} om at kunne se et katalog over projekter og vejledere. Og igennem samme \textit{Models} opfylder vi endnu et \textit{brugsscenarie} om at kunne præsentere \textit{up-to-date} oplysninger omkring vejledere, projekter, publikationer og kontaktoplysninger. \\ \\
Vi er derudover kommet frem til, at vi vil verificerer personer som optræder på siden, med deres email-addresse. Dette mener vi er den bedste løsning, da vi tager email-addresserne fra diku's hjemmeside, så antager vi kun en bruger for at være verificeret hvis vi kan se den email-addresse stående under en person på diku's hjemmeside. Vi lader kun verificerede brugere ændre i projekter og andet. Derudover må en verificeret bruger selvfølgelig kun ændre på sine egne data.
\section{Afprøvning}
Vi har til denne rapport lavet testing af følgende:
\begin{itemize}
\item Django webapplikation
	\begin{itemize}
	\item Bestående af \textit{Models} og en \textit{MySQL-database} har vi indsat værdier og vist dem på vores webserver.
	\end{itemize}
\item Scraper
	\begin{itemize}
	\item Vi har kørt vores scraper på diku's hjemmeside over ansatte og printet resultatet.
	\end{itemize}
\item Controller
	\begin{itemize}
	\item Controller er vores eget lille modul, som er designet til at lette vores arbejde ved implementation af f.eks. Scraperen. Vi har testet flere af funktionerne igennem \textit{unit-testing}.
	\end{itemize}
\end{itemize}
\subsection{Django webapp}
\subsection{Scraper}
Til denne test har vi kørt vores scraper på hjemmesiden over DIKU Ansatte: \url{http://www.diku.dk/Ansatte}, og høster information baseret på nogle foruddefinerede HTML-tags. Nedenstående er et udsnit af vores resultat. \\ \\
Scraperen gemmer endnu ikke noget information i nogen database, men sender den høstede information til stdout. Et eksempel på programmets output kan ses herunder:
\begin{lstlisting}
Name: Abelskov, Hjalte
Link: http://diku.dk/Ansatte?pure=da/persons/432412
None
Instruktor


None
Name: Abrahamsen, Mikkel
Link: http://diku.dk/Ansatte?pure=da/persons/289414
None
ph.d.-studerende


Name: E-mail
Link: http://diku.dk/Ansatte
None
Name: Adamaszek, Anna Maria
Link: http://diku.dk/Ansatte?pure=da/persons/506000
None
Postdoc
\end{lstlisting}
\subsection{Controller}
Testene er placeret i vores test-mappe, herinde har vi lavet 7 forskellige tests, som har til formål at vise om testene bliver klaret, hvad deres output er, og hvor hurtig testene er. Testene er beskrevet nedenfor:
\begin{itemize}
\item \textit{files\_test()}
  \begin{itemize}
  \item Beskrivelse: Laver en mappe og 3 filer, derefter bliver indholdet af filerne printet.
  \item Forventet output: Stringene som er blevet skrevet til filerne. En lang string, og to små.
  \end{itemize}
\item \textit{folder\_test()}
  \begin{itemize}
  \item Beskrivelse: Laver en mappe kaldet \textit{test}, og placerer 3 mapper inden i denne mappe. Printer derefter alle undermappe i mappen \textit{test}.
  \item Forventet output: 3 mapper.
  \end{itemize}
\item \textit{last\_modified\_test()}
  \begin{itemize}
  \item Beskrivelse: Laver 3 filer, hvoraf den ene først bliver lavet med 1 sekunders \textit{delay}, derefter bliver deres last modified timestamp printet.
  \item Forventet output: Deres tid siden epoch, og et convertet læseligt timestamp. Den ene skal være anderledes grundet delay.
  \end{itemize}
\item \textit{last\_modified\_comparison\_test()}
  \begin{itemize}
  \item Beskrivelse: Sammen som overstående, men deres timestamps bliver sammenlignet.
  \item Forventet output: Mappe 1 og 2 giver \textit{True}, mens mappe 1 og 3 giver \textit{False}.
  \end{itemize}
\item \textit{files\_ext\_test()}
  \begin{itemize}
  \item Beskrivelse: Laver 4 filer, hvoraf den ene har et andet format. Derefter bliver alle filerne med den ene type printet, og derefter filer med den anden type.
  \item Forventet output: Først skriver den alle \textit{.txt} og derefter den ene \textit{.dat}.
  \end{itemize}
\item \textit{clear\_file\_test()}
  \begin{itemize}
  \item Beskrivelse: Laver en fil og skriver en \textit{String} til den. Printer indeholdet, derefter sletter den alt i filen og printer dens nye indhold.
  \item Forventet output: Først har filen et indhold der bliver printet, og derefter intet indhold.
  \end{itemize}
\item \textit{socket\_test()}
  \begin{itemize}
  \item Beskrivelse: Opretter en socket, og prøver at hente indholdet af websiden example.org. Skriver indeholdet til en fil, som derefter bliver printet.
  \item Forventet output: Vi får en HTTP response tilbage, og derefter sidens html-kode.
  \end{itemize}
\end{itemize}
Testene bliver kørt en efter hinanden, før en test bliver kørt begynder vi at tage tid, testen bliver kørt og efter testen er færdig bliver tiden printet. Efter hver test er der 1 sekunders pause, for at gøre det lettere at læse outputtet, inden den næste test bliver kørt.
\subsubsection{Output}

Vores tests er opbygget sådan, at vis testene ikke fejler så er alt gået godt. Derfor hvis vi ikke for nogle errors fra python, og hvis outputtet er korrekt så er alt gået godt.
Under vises et udsnit af test resultaterne:
\begin{verbatim}
Starting last modified comparison test!
wut.txt last modified converted: Tue, 15 March 2016 10:39:02 GMT
wat.txt last modified converted: Tue, 15 March 2016 10:39:02 GMT
wot.txt last modified converted: Tue, 15 March 2016 10:39:03 GMT
wut.txt == wat.txt?: True
wut.txt == wot.txt?: False
Last modified comparison test took: --- 1.00246596336 seconds ---

Starting files extention test!
Creating files wut.txt, wat.txt, wot.txt, tad.dat
List of all .txt files in folder test: ['wot.txt', 'wat.txt', 'wut.txt']
List of all .dat files in folder test: ['tad.dat']
Files extension test took: --- 0.00122594833374 seconds ---

Starting socket test!
Contents of request response: HTTP/1.1 200 OK
Accept-Ranges: bytes
Cache-Control: max-age=604800
Content-Type: text/html
Date: Tue, 15 Mar 2016 09:39:07 GMT
Etag: "359670651"
Expires: Tue, 22 Mar 2016 09:39:07 GMT
Last-Modified: Fri, 09 Aug 2013 23:54:35 GMT
Server: ECS (iad/18F0)
Vary: Accept-Encoding
X-Cache: HIT
x-ec-custom-error: 1
Content-Length: 1270
Connection: close
Socket test took: --- 0.243407011032 seconds ---

\end{verbatim}

Vores tests er opbygget på sådan en måde, at de bare returner en \textit{Boolean}, så vi ved om testen er fejlet eller ej. \\ \\
Under vises et udsnit af test resultaterne:
\begin{lstlisting}
Starting files test!
Files test returned: True
Files test took: --- 0.000173807144165 seconds ---

Starting folder test!
Folder test returned: True
Folder test took: --- 0.000287055969238 seconds ---

Starting last modified test!
Last modified test returned: True
Last modified test took: --- 1.00135111809 seconds ---
\end{lstlisting}




\section{Design}
Django benytter \textit{models}, hver model knytter sig til en enkelt tabel i den tilhørende database. En model i Django er en Python klasse. \\ I vores projekt benytter vi indtil videre klasserne \texttt{Supervisor} og \texttt{Project}. På Figur \ref{fig:modelUML} ses et UML diagram for klasserne \texttt{Supervisor} og \texttt{Project}.


\begin{figure}
			\centering
			\includegraphics[width=1.0\textwidth]{modelUML.pdf}
			\caption{}
			\label{fig:modelUML}
\end{figure}



\section{Planlægning af næste iteration}
Vi vil fokusere på tre \textit{use cases} i næste iteration. Disse \textit{use cases} er,
\subsection*{Forventes implementeret i næste iteration}
\begin{enumerate}
	\item \label{enum:katalog} Det skal være muligt at se et katalog over projekter.

	\item \label{enum:verifikation} Personer, der optræder på siden, skal kunne verificeres, hvis de eksempelvis ønsker at tilføje projekter eller foretage rettelser.

\subsection*{Påbegyndes i næste iteration}

	\item \label{enum:vejlederInfo} Systemet skal kunne præsentere \textit{up-to-date} oplysninger om potentielle vejledere, herunder:
	\begin{itemize}
		\item Tidligere projekter.
		\item Forskningspublikationer.
		\item Kontaktoplysninger.
	\end{itemize}
\end{enumerate}

\subsection{Underopgaver for de udvalgte \textit{use cases}}
\subsubsection*{\textit{Use case} \ref{enum:katalog}}
\begin{itemize}
	\item Django \textit{modellen} for et projekt skal færdiggøres.
	\item Der skal laves en \textit{template} knyttet til denne model.
\end{itemize}

\subsubsection*{\textit{Use case} \ref{enum:verifikation}}
De specifikke detaljer, for verificeringen, skal overvejes og konkretiseres nærmere. Verificeringen skal foregå med mindst mulig adminstrativ vedligholdelse, men samtidig korrekt. Hvi den deles op i underopgaver er disse,
\begin{itemize}
	\item Der skal en password generering.
	\item Muligheden for oprettelse af en bruger, med prædefinerede valg, ud fra det høstede data.
\end{itemize}

\subsubsection*{\textit{Use case} \ref{enum:vejlederInfo}}
Den høstede information om vejlederne skal benyttes til automatisk at oprette brugere i Django. Hver af disse brugere skal herefter kunne overtages de respektive vejledere.







\end{document}
