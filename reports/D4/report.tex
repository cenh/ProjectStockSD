\documentclass[11pt]{article}
\usepackage[a4paper, hmargin={2.8cm, 2.8cm}, vmargin={2.5cm, 2.5cm}]{geometry}
\usepackage{eso-pic} % \AddToShipoutPicture
\usepackage{graphicx} % \includegraphics
\usepackage[utf8]{inputenc}
\usepackage[danish]{babel}
\usepackage[T1]{fontenc}
\usepackage{hyperref}
\usepackage{nameref}
\usepackage{amsmath, amscd}
\usepackage{amsmath,amscd}
\usepackage{amssymb}
\usepackage{amsthm}
\usepackage{enumerate}
\usepackage{graphicx}
\usepackage{framed}
\usepackage{color}
\usepackage{listings}
\usepackage{float}
\lstset{
	frame=single,
	breaklines=true,
	postbreak=\raisebox{0ex}[0ex][0ex]{\ensuremath{\color{red}\hookrightarrow\space}}
}
%% Change `ku-farve` to `nat-farve` to use SCIENCE's old colors or
%% `natbio-farve` to use SCIENCE's new colors and logo.
\def \ColourPDF {../include/ku-farve}

%% Change `ku-en` to `nat-en` to use the `Faculty of Science` header
\def \TitlePDF {../include/ku-en}  % University of Copenhagen

\title{
  \vspace{3cm}
  \Huge{D4} \\
  \Large{Software Udvikling 2016}
}

\author{
	\Large{Stefan Friis Tofte} - \textbf{jwr342}% - \texttt{stefan.f.tofte@gmail.com}
	\and
	\Large{Mads Kronborg} - \textbf{xlq446}% - \texttt{kronborg96@gmail.com}
	\and
	\Large{Lasse Halberg Haarbye} - \textbf{lpt113}% - \texttt{ninjalf2@gmail.com}
	\and
	\Large{Christian E.N. Hansen} - \textbf{vmk541}% - \texttt{cralle@outlook.com}
}

\begin{document}


\AddToShipoutPicture*{\put(0,0){\includegraphics*[viewport=0 0 700 600]{\ColourPDF}}}
\AddToShipoutPicture*{\put(0,602){\includegraphics*[viewport=0 600 700 1600]{\ColourPDF}}}

\AddToShipoutPicture*{\put(0,0){\includegraphics*{\TitlePDF}}}

\clearpage\maketitle
\thispagestyle{empty}

\newpage
\tableofcontents
\newpage

\section{Ændrede krav og implementering af opgaver}
Kravene til vores produkt er ikke blevet ændret ift. til de \textit{use cases}, der blev specificeret i starten af forløbet. Disse \textit{use cases} blev beskrevet i delaflevering D1. \\
Koordinering af arbejdet i denne iteration er foregået som hidtil. Vi har talt sammen om opgaverne ved de ugentlige møder, samt oprettet \textit{issuses} på Github.

\section{Afprøving}
For at foretage \textit{acceptance testing} af vores produkt har vi benyttet et \textit{acceptance test framework}. Vi har valgt \textit{'Robot framework'}, og til dette \textit{framework} benytter vi biblioteket \textit{'Selinium2Library'}. \\
Vi har hørt om dette \textit{framework} ved workshop nr. 2, der handlede om værktøjer.

\subsection{Testresultater}

\section{Refaktorering}
Vi foretaget en omstrukturering af vores mappehieraki for vores \textit{Django}-projekt. \textit{Templates}, bruges i \textit{Django} til at danne HTML-sider. Mappe der indeholder \textit{templates} er blevet omstruktureret således at hver model har fået en mappe, der indeholder de tilhørende \texttt{.html}-filer. \\
Derudover er der foretaget en ændring af navnene på nogle af klasser, funktioner og variable, for at gøre navngivningen mere ensartet.

\section{Fremtidige planer}
For at opfylde de \textit{use cases}, der blev opstillet i starten af forløbet, har vi oprettet \textit{issuses} på Github. Disse \textit{issues} beskriver nogle af de problemer, der skal løses for at kunne opfylde kravene for produktet. \\
Disse \textit{issues} har fået forskellige \textit{keywords}, der beskriver hvilke kategorier de tilhører:
\begin{itemize}
	\item{design}
	\item{enhancement}
	\item{database/models}
	\item{scraper}
	\item{testing}
\end{itemize}
En start kunne derfor være at for løst disse \textit{issues}. \\
På nuværende tidspunkt er de vigtigste uløste \textit{issues}: 'Registrering af brugere' og 'Rettigheder for brugere'. For at uddybe, hvad der mere konkret menes med disse \textit{issue}-titler referes her til denne \textit{use case} for produktet:
\begin{itemize}
 \item[]{Personer, der optræder på siden, skal kunne verificeres, hvis de eksempelvis ønsker at tilføje projekter eller foretage rettelser.}
\end{itemize}
Der er dele af koden, der ikke er unittestet, her ville de være optimalt, hvis skrevet unittests for alle klasser og funktioner.
\end{document}
