\documentclass[12pt]{article}
\usepackage[a4paper, hmargin={2.5cm, 2.5cm}, vmargin={2.5cm, 2.5cm}]{geometry}
\linespread{1.5}
\usepackage{eso-pic} % \AddToShipoutPicture
\usepackage{graphicx} % \includegraphics
\usepackage[utf8]{inputenc}
\usepackage[danish]{babel}
\usepackage[T1]{fontenc}
\usepackage{hyperref}
\usepackage{amsmath, amscd}
\usepackage{amsmath,amscd}
\usepackage{amssymb}
\usepackage{amsthm}
\usepackage{enumerate}
\usepackage{graphicx}
\usepackage{framed}
\usepackage{color}
\usepackage{listings}
\usepackage{float}
\lstset{
	frame=single,
	breaklines=true,
	postbreak=\raisebox{0ex}[0ex][0ex]{\ensuremath{\color{red}\hookrightarrow\space}}
}
%% Change `ku-farve` to `nat-farve` to use SCIENCE's old colors or
%% `natbio-farve` to use SCIENCE's new colors and logo.
\def \ColourPDF {../include/ku-farve}

%% Change `ku-en` to `nat-en` to use the `Faculty of Science` header
\def \TitlePDF {../include/ku-en}  % University of Copenhagen

\title{
  \vspace{3cm}
  \Huge{Eksamensrapport} \\
  \Large{Software Udvikling 2016 - Project-Stock}
}

\author{
	\Large{Stefan Friis Tofte} - \textbf{jwr342} - \texttt{stefan.f.tofte@gmail.com}
	\and
	\Large{Mads Kronborg} - \textbf{xlq446} - \texttt{kronborg96@gmail.com}
	\and
	\Large{Lasse Halberg Haarbye} - \textbf{lpt113} - \texttt{ninjalf2@gmail.com}
	\and
	\Large{Christian E.N. Hansen} - \textbf{vmk541} - \texttt{cralle@outlook.com}
}

\begin{document}


\AddToShipoutPicture*{\put(0,0){\includegraphics*[viewport=0 0 700 600]{\ColourPDF}}}
\AddToShipoutPicture*{\put(0,602){\includegraphics*[viewport=0 600 700 1600]{\ColourPDF}}}

\AddToShipoutPicture*{\put(0,0){\includegraphics*{\TitlePDF}}}

\clearpage\maketitle
\thispagestyle{empty}

\newpage
\tableofcontents
\newpage

\section{Indledning}
\label{sec:indledning}
% Christian

\section{Problembeskrivelse}
\label{sec:problem}
% Mads

\section{Krav}
I forløbets start opstillede vi nogle \textit{use cases}, for at imødekomme de krav vores \textit{product owner} havde til det produkt vi skulle udvikle. \\
Vi foretog en løs priotering af produktets \textit{use cases}. Vi vurderede, hvad der var kernefunktionaliteter, og hvad der var \textit{'nice-to-have'}-funktionaliteter.

\subsection{\textit{Use cases}}
\label{sec:krav}

\subsubsection{Kernefunktionalitetter}
\begin{enumerate}
	\item Det skal være muligt at se et katalog over projekter.
	\item Systemet skal kunne præsentere \textit{up-to-date} oplysninger om potentielle vejledere, herunder:
	\begin{itemize}
		\item Tidligere projekter.
		\item Forskningspublikationer.
		\item Kontaktoplysninger.
	\end{itemize}
	\item Personer, der optræder på siden, skal kunne verificeres, hvis de eksempelvis ønsker at tilføje projekter eller foretage rettelser.
\end{enumerate}

\subsubsection{\textit{'Nice-to-have'}-funktionaliteter}
\label{sec:krav}
% Stefan
I forløbets start opstillede vi nogle \textit{use cases}, for at imødekomme de krav vores \textit{product owner} havde til det produkt vi skulle udvikle.

\subsection{\textit{Use cases}}
\begin{enumerate}
  \item Systemet skal kunne vise projekter knyttet til specifikke \textit{keywords}.

  \item Systemet skal vise en status på en vejleders arbejdsbyrde \textit{up-to-date} (hvad der præcist menes med dette, skal yderlige specificeres).
  \item En vejleder eller en person knyttet til en \textit{business-club}, skal kunne tilføje et projekt til projektkataloget.

	\item Systemet skal kunne filtrere vejledere i forhold til \textit{keyswords} i forskningspublikationer (titler).
\end{enumerate}


\section{Design}
\label{sec:design}
% Lasse

\section{Afprøvning}
\label{sec:afproevning}
% Mads

\section{Udviklingsmiljø}
\label{sec:udvikling}
% Christian

\section{Diskussion og reflektion}
\label{sec:diskussion}
% Alle

\section{Konklusion}
\label{sec:konklusion}
% Christian

\section{Bilag}
\label{sec:bilag}

\subsection{Bilag A}
\label{sec:bilagA}
% Stefan

\subsection{Bilag B}
\label{sec:bilagB}
% Lasse

\subsection{Bilag C}
\label{sec:bilagC}
% Stefan

\end{document}
